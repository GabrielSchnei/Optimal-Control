\documentclass[12pt,a4paper,oneside]{scrartcl}

\usepackage[utf8]{inputenc}
\usepackage[T1]{fontenc}
\usepackage{lmodern}
\usepackage{tcolorbox}
\usepackage{microtype}
\usepackage{geometry}
\geometry{left=3cm,right=2.5cm,top=3cm,bottom=3cm}
\usepackage{setspace}
\onehalfspacing

\usepackage{amsmath,amssymb,amsthm,mathtools}
\usepackage{bm}
\usepackage{dsfont}
\usepackage{physics}
\usepackage{nicefrac}
\usepackage{enumitem}
\setlist[itemize]{topsep=0pt, itemsep=2pt}

\usepackage{tikz}
\usepackage{pgfplots}
\pgfplotsset{compat=1.17}
\usepackage{graphicx}
\usepackage{caption}
\usepackage{subcaption}
\usepackage{etoolbox}

% Theorem
\newtheorem{lemma}{Lemma}[section]
\newtheorem{definition}{Definition}[section]

\newenvironment{lemmabox}[1][]{
	\refstepcounter{lemma}
	\begin{tcolorbox}[colback=blue!5!white,
		colframe=blue!60!black,
		fonttitle=\bfseries,
		boxrule=0.8pt,
		arc=3pt,
		left=8pt,right=8pt,top=6pt,bottom=6pt,
		title={Lemma \thelemma},
		#1]
		}{
	\end{tcolorbox}
}

\newenvironment{definitionbox}[1][]{
\refstepcounter{definition}
\begin{tcolorbox}[colback=blue!5!white,
	colframe=blue!60!black,
	fonttitle=\bfseries,
	boxrule=0.8pt,
	arc=3pt,
	left=8pt,right=8pt,top=6pt,bottom=6pt,
	title={Definition \thedefinition},
	#1]
}{
\end{tcolorbox}
}

% Kopf- und Fußzeile
\usepackage{scrlayer-scrpage}
\clearpairofpagestyles
\ihead{Optimal Control}
\ohead{WS 2025/26}
\cfoot{\pagemark}
\pagestyle{scrheadings}
\setcounter{section}{-1}

\begin{document}
	
	% Titelseite
	\begin{titlepage}
		\centering
		\vspace*{3cm}
		{\Huge\bfseries Optimal Control}\par
		\vspace{1.5cm}
		{\Large Wintersemester 2025/26}\par
		\vspace{0.5cm}
		{\Large Dozent: Prof. Dr. Andrea Iannelli}\par
		\vfill
		{\today}
	\end{titlepage}
	
	\pagenumbering{roman}
	\tableofcontents
	\clearpage
	\pagenumbering{arabic}
	
	\section{Introduction}
	\[
	\dot{x} = f(t,x,u), \quad x(t_0)=x_0, \quad t \in [t_0,t_f]
	\]
	\[
	f : [t_0,t_f]\times\mathbb{R}^{n_x}\times\mathbb{R}^{n_u} \to \mathbb{R}^{n_x}
	\]
	\[
	x = \text{state}, \quad u = \text{input}
	\]
	
	Initial Value Problem (IVP)
	
	Given $x_0, u(\cdot)$ we can compute $x(\cdot)$ \\
	\hspace*{23.5mm}$\rotatebox[origin=c]{90}{$\Rsh$}$ functions of time $\rotatebox[origin=c]{270}{$\Lsh$}$
	
	When is this possible? It depends on $f$.
	
	
	\begin{lemmabox}
		\textbf{(Sufficient conditions)}\\
		Existence \& Uniqueness of solutions of ODEs.\\
		Assume that
		\begin{itemize}[]
			\item $f$ is piecewise continuous in $t$ and $u$
			\item $f$ is globally Lipschitz in $x$
			\[
			\exists\, k(t,u)\, \text{ s.t. } \|f(t,x_1,u)-f(t,x_2,u)\|\le k(t,u)\|x_1-x_2\|,\ \forall x_1,x_2 \in \mathbb{R}^{n_x}
			\]
		\end{itemize}
		Then $x(\cdot)$ exists for all $t$ and is unique.
	\end{lemmabox}
	
	\subsection*{Remarks}
	\begin{itemize}[noitemsep]
		\item Lipschitz continuous $\Rightarrow$ continuous, but not the converse
		\item $\sqrt{x}$ is continuous but not Lipschitz, $\dot x = \sqrt{x}$ does not have a unique solution
		\item Continously differentiable $(\mathcal{C}^1)$ $\Rightarrow$ locally Lipschitz continous $\forall x_1,x_2 \in \mathcal{X} \subset \mathbb{R}^{n_x}$
		\item Locally Lipschitz continuous x guarantees existence \& uniqueness for small enough times
	\end{itemize}
	
	In this course we will assume $f \in\mathcal{C}^1$ and implicitely assume that $t_f$ is chosen such that $x(\cdot)$ exists in $[t_0,t_f]$. \\\\
	We do not need to worry about existence \& uniqueness!
	
	\subsection*{Goal in Optimal Control:}
	Design $u$ such that
	\begin{enumerate}
		\item $u(t) \in \underset{\uparrow}{\mathcal{U}(t)}$, $x(t) \in \underset{\uparrow}{\mathcal{X}(t)} \quad \forall t \in [t_0,t_f], \quad \mathcal{X}\subset\mathbb{R}^{n_x}, \ \mathcal{U}\subset\mathbb{R}^{n_u}$\\
			sets defining constraints on $u \& x$ \\
			\hspace*{10mm}$\Rightarrow$ Admissible input/state trajectories
		\item
			The system behaves optimally according to
			\[
			\underset{\uparrow}{J}(u) = \int_{t_0}^{t_f} \underset{\uparrow}{l}(t,x(t),u(t))\,dt + \underset{\uparrow}{\varphi}(t_f,x(t_f))
			\]
			\hspace*{18mm}Cost function\hspace*{10mm}running cost\hspace*{10mm}terminal cost\\
			\hspace*{10mm}$\Rightarrow$ optimal behaviour
	\end{enumerate}
	
	Formally, we can state the goal as follows: \\
	Find an admissible input $u^\star$ which causes the dynamics to follow an admissible trajectory $x\star$ which minimizes $J$, that is
	\[
	\int_{t_0}^{t_f} l(t,x^\star(t),u^\star(t))\,dt + \varphi(t_f,x^\star(t_f)) \leq \int_{t_0}^{t_f} l(t,x(t),u(t))\,dt + \varphi(t_f,x(t_f))
	\]
	\hspace*{100mm}$\forall \text{ admissible } x,u$
	
	\subsection*{Examples of cost functions}
	\begin{enumerate}[label=\arabic*)]
		\item Minimum-time problem\\
		Goal: transfer the system from $x_0$ to a set $\mathcal{S}$ in the minimum time
		\[
		J = t_f-t_0 = \int_{t_0}^{t_f}dt \qquad (l=1, \varphi=0)
		\]
		\[
		x(t_f) \in \mathcal{S}
		\]
		Note: $t_f$ is also a decision variable! The unknowns are $(u,t_f)$.
		
		\item Minimum control-effort problem
		\[
		J = \int_{t_0}^{t_f} \|u(t)\|^2 \, dt
		\]
		\[
		x(t_f) \in \mathcal{S}
		\]
		
		\item Tracking problem
		\[
		J = \int_{t_0}^{t_f} (x(t)-r(t))^T Q (x(t)-r(t))dt
		\]
		$Q > 0$ (positive definit matrix: symmetric \& all eigenvalues positive) \\
		$r(t)$ given signal
	\end{enumerate}
	
	\section{Nonlinear Programming}
	
	Nonlinear Programs (NLP) are general \underline{finite-dimensional} optimization problems:
	\[
	\underset{x}{\min} f(x)
	\]
	\[
	\text{s.t. } g(x)\leq0, \quad h(x)=0
	\]
	$f: \mathbb{R}^n \rightarrow \mathbb{R}$, objective function\\
	$g: \mathbb{R}^n \rightarrow \mathbb{R}^{n_g}$, inequality constraints\\
	$h: \mathbb{R}^n \rightarrow \mathbb{R}^{n_h}$, equality constraints\\
	Feasible set:
	\[
	D = \{x\in\mathbb{R}^n \mid g(x)\leq0,\ h(x)=0\}
	\]
	$\overline{x}\in D$ feasible point
	
	\begin{definitionbox}
		\textbf{Global, local Minimizers}\\
			$x^\star\in \mathcal{D}$ \underline{Global Minimizer} of the NLP if
			\[
			f(x^\star)\le f(x)\quad \forall x\in \mathcal{D}
			\]
			$f(x^\star)$ is the \underline{Global Minimum} (or Minimum)\\
			Nomenclature: $x^\star$ is also called (optimal) solution, $F(x^\star)$ is optimal value\\
			$x^\star$ is a strict global minimizer if $f(x^\star)<f(x)\quad \forall x \in\mathcal{D}$\\
			$x^\star\in \mathcal{D}$ \underline{Local Minimizer} if
			\[
			\exists \varepsilon>0,\text{ s.t.}\ f(x^\star)\le f(x)\quad \forall x\in B_\varepsilon(x^\star)\cap \mathcal{D}
			\]
			\[
			B_\varepsilon(x) := \{ y \mid \|x-y\|\le\varepsilon\} \qquad \|\cdot\|:\mathbb{R}^n\rightarrow\mathbb{R}_{\geq 0}\text{ any norm in }\mathbb{R}^n
			\]
			Strict local Minimizer if inequality holds strictly\\
			Global min $\underset{\not\leftarrow}{\overset{\rightarrow}{}}$ local min
	\end{definitionbox}
	
	Solving an NLP boils down to finding global or local minimizers. \\
	Does a solution always exist? No.
	
\end{document}
